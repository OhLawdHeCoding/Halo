\section{Discussion}
\subsection{Reflection}
The main obstacle we encountered during the project was the new technology. Since none of us had used the technology we used before, estimating tasks accurately was difficult. Oftentimes a task that was delegated a very high amount of points turned out to be trivial upon closer inspection. Similarly, tasks that were delegated a small amount of points turned out to be quite difficult. We were very quick to make our estimations. Had we been more methodical and researched our tasks more before making our estimations, they would have been more accurate. 

Of course, now that we have completed the project and we are all more experienced with the technologies that we used our estimations would have been far more accurate. But this is not exclusive to the agile process and could be said about anything. When you are learning something, the more you learn about that thing, the easier that thing becomes. So this may be a bit of a moot point. 

A more sophisticated system for handling issues would be appropriate as well. There were some issues that arose that were hard to fix. When these issues occurred, team members would have to ask other teams for help but since there was no system for describing and tracking these issues, it was harder than it should have been to reach out to other teams, describe the problem, and ask for help. We are aware that there are tools to handle this issue, such as GitHub Issues, but no such tool was used. 

The weekly meetings on Sundays where we did both the team reflections and the sprint reviews worked well for us, as opposed to having the meeting right after the sprint on Fridays evenings. It allowed us to reflect on what we’d achieved and gave us time to figure out new ideas for future implementations that we then could present for the group during the meetings. However, since it was a lot to go through in only one sitting, to have two or three shorter meetings might have made each meeting more energetic. The 3rd party-review of our project was very helpful and made us reconsider several design choices and also resulted in some new features being added. As mentioned previously some of these were not implemented but it started a discussion which was healthy. To have someone outside the working team give feedback is valuable and, looking back, it might have been rewarding to do more than once.

Regarding our use of product owners, we started by being P.O’s of our own tasks which resulted in very few questions from the rest of the group about our design choices. This was mostly for the better as it allowed us to really get into it fast which was probably necessary with the course not having so many sprints. Too many questions during the first couple of weeks could potentially lead to an unfinished product or stress later on during the project. However, after the first week, we decided to separate product owners from developers to allow for more discussion, which overall most definitely led to better code, when other people gave their opinion on your work.


\subsection{What We Have Learned As a Group}
Throughout the course of the project, the team as a whole has picked up both tool and workstyle related skills. To begin with, the team members have gained concrete skills in common programming paradigms such as JavaScript\cite{JavaScript} and the subsidiary framework “React”\cite{React}, as well as the related CSS\cite{CSS} and HTML\cite{HTML} languages. Furthermore, (although previously experienced) skills were improved with GIT and Github which were used to handle version control and host the repository. 

In the realm of more abstract skills, the team learned about the Agile work-method*. More specifically, Scrum was the Agile methodology practiced throughout the project (although with incremental and team influenced modifications). One of these modifications was originally a misunderstanding of the Scrum framework, and resulted in us defining intra-sprint MVPs. However, once this discovery was made our “sprint MVPs” had already become a beloved part of our planning process which allowed us to easily prioritize user stories. This had a lot of synergy with the agile mindset, as it led to the most integral features getting completed early in the sprint, allowing them to potentially be reviewed by a PO and revised before the official sprint review. Moreover, the “sprint MVPs” in and of themselves were a convenient summary of sprint-by-sprint progress. This aspect is something that we believe would provide value for investors/overseers not concerned with intra-sprint details, but with a birds eye view* of the progress. 

Concerning creating a good birds eye view of the project, we chose three KPIs to keep track of our progress. These KPIs were “Burndown”, “Velocity” and “Code Churn”, and are all productivity focused. Our intention was to focus on productivity to quickly get the project started, and change KPIs later if needed (in alignment with the agile work method). However, in the last half of the project, we realised the importance of diversifying the KPIs as we better understood the considerable overlap in our current lineup. We consequently added another metric, customer feedback, which quickly became the most useful datapoint in describing the state of the project in the opinion of the team. Furthermore, we learned the importance of KPI-planning* in the initial project planning stage and that it should not be glossed over to, for instance, avoid bad data points. For example, with “Code Churn” we used gitHub’s built-in* statistics page. However, it only considered commits directly to main (excluding even merges) which meant that team members with the preference of creating git branches were invisible to the KPI. 


\subsection{Goals For Future Endevours}
In a future agile project it may be useful to perform more research regarding the implementation of various features to ensure that our estimated effort for the tasks are more accurate. Likely, doing so will help in dividing tasks more evenly among the team members and thereby increase productivity.

One of the most useful tools would be to implement a system that assists people when they need help with an issue. At times during the project people requested assistance without receiving any. The problem partly stems from a lack of clarity. To request help one had to post a question in the discord chat. Whether or not the request was answered was unclear to other team participants and there was no clear structure that ensured help was given. In a future agile project this issue can be resolved by utilizing existing tools like the issue tracker in GitHub. Moreover, assigning a weekly meet up where the entire team works together and advises one another may move the project along faster and prevent pauses in productivity.

The weekly meetings for sprint reviews would be implemented again. To have these meetings a couple days after the end of a sprint worked well and could be further utilized by officially requesting that the team reflect on future improvements once they complete the sprint. Looking over the project at the end of a sprint may improve both the understanding of the current implementation as well as result in more creative ideas regarding future development. To an extent this was already done but making it an official task every sprint may yield further improved productivity. 

P.Os would be further utilized as they were very useful. Having a person to answer to regarding one's work is very important as it incentivises a better work ethic and promotes useful and understandable features.

Having third party opinions (opinions from the customer) helped create a website that actually serves the customer. This will be useful in most agile or non agile projects as their goal is to deliver value to the customer.

Pair programming...


I implementeringen störenheten används en slumpgenerator från C Standard General 
Utilities Library . %http://www.cplusplus.com/reference/cstdlib/rand/